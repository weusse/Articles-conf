\section{Introduction}\label{Introduction}
Malaria is a life-threatening disease caused by parasites that are transmitted to people through the bites of infected female Anopheles mosquitoes. It is preventable and curable. Malaria is an acute febrile illness. In a non-immune individual, symptoms usually appear 10–15 days after the infective mosquito bite. The first symptoms – fever, headache, and chills – may be mild and difficult to recognize as malaria. If not treated within 24 hours, P. falciparum malaria can progress to severe illness, often leading to death. Children with severe malaria frequently develop one or more of the following symptoms: severe anaemia, respiratory distress in relation to metabolic acidosis, or cerebral malaria. In adults, multi-organ failure is also frequent. In malaria endemic areas, people may develop partial immunity, allowing asymptomatic infections to occur. In 2019, there were an estimated 229 million cases of malaria worldwide. The estimated number of malaria deaths stood at 409 000 in 2019. The WHO African Region carries a disproportionately high share of the global malaria burden. In 2019, the region was home to 94\% of malaria cases and deaths, thanks to the  2019 World Malaria Report \cite{19WMR}. 
Over the past years, many efforts have been done by governmental and non governmental organizations  to eradicate Malaria:  actions continuously conducted by the WHO are real examples of those.  In the research field, many studies, aiming at understanding the disease from the Plasmodium mosquito point of view or proposing automated detection tools, have been conducted \cite{Ga19,Le74,ermert2011development,Hu17}. The Rapid Diagnostic Test (RDT) \cite{Hu17} is one of the most successful and prominent introduced tool to automatically predict whether or not a given patient suffers from Malaria. It relies on the
detection of specific Plasmodium proteins, PfHRP2, pLDH
and aldolase. The RDT is largely used and adopted as a standard in many health structures in Sub-African countries because of its simplicity to utilize and does not require any specific domain knowledge. However as highlighted in \cite{Hu17} the RDT is not fully reliable:  in Section \ref{Methods} we show that the precision of the RDT is about 90\% for the real datasets used in this study.
With the development and increasing adoption of automated tools in the health field, machine learning  (ML) \cite{mitchell1997machine, Ug1} applications might help medical actors in their decision-making process. 
 In this paper, we propose an  extensive comparative study of six machine learning algorithms, among the most popular for the prediction of Maria in Senegal. The evaluated and compared ML algorithms are Naive Bayes (NB), Logistic Regression(LR) ,  Decision Tree(DT), Support Vector Machine(SVM) ,
 Random Forest(RF),
 and Artificial Neural Network(ANN). We conducted experiments on five datasets based on the two real world datasets about Senegalese citizens that suffer or not from Malaria. These two datasets have been collected in two different contexts and contain clinical data such as sign, symptom and final diagnostic of patients living in distinct locations in Senegal (for the first dataset) or within the same area (for the second dataset). Those patients have been examined by doctors in given health services and their clinical data recorded: for each patient the final diagnostic is provided with the corresponding signs and symptoms. The outcome of the RDT is also provided. 
% paper organization
The rest of the paper is structured as follows.First we gives detail explanation of machine learning metthods and the different data sets methods used for this study in section 2. In section 3 our  differents results are presented. Section 4 and 5 includes discussion of our results and conclusions of this paper.
 \newpage