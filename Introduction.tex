\section{Introduction}\label{Introduction}
Malaria is a transmissible disease through the bites of infected female Anopheles mosquitoes. It comes with symptoms such as fever, headache, and chills in its early stage and can evolve to more severe health problems (severe anaemia, respiratory distress, etc.) often leading death. In 2019, the number of Malaria cases worldwide has been estimated to 229 millions. The number of deaths caused by Malaria has been approximatively estimated to 409 000 in 2019; the African area represents around 94\% of the reported malaria cases and deaths in 2019, thanks to the annual world Malaria report \cite{19WMR}. 

Over the past years, many efforts have been made by governmental and non governmental organizations (e.g. WHO) to eradicate Malaria in the world.  In the research field, many studies, aiming at understanding the disease from the Plasmodium mosquito point of view or proposing automated detection tools, have been conducted \cite{Ga19,Le74,ermert2011development,Hu17}. The Rapid Diagnostic Test (RDT) \cite{Hu17} is one of the most successful and prominent introduced tool to automatically predict whether or not a given patient suffers from Malaria. It relies on the detection of the presence of specific Plasmodium proteins, PfHRP2, pLDH
and aldolase in human blood. The RDT is largely used and adopted as a standard many Sub-saharan African countries such as Senegal. However, as proved in \cite{Hu17}, RDT is not fully reliable:  Section \ref{Methods} shows that the precision of RDT is about 90\% for datasets used in this study. Despite those advanced tools, Malaria is still a real public health in sub-Saharan African countries such as Senegal because of the lack of appropriate care support or late and error-prone detection of the disease.
Artificial intelligence is now recognized as a domain that may help medical actors in their decision-making process. \cite{mitchell1997machine, Ug1}  
 This paper proposes an  extensive comparative study of the most popular machine learning models for the task of Malaria prediction. The evaluated and compared ML algorithms are Naive Bayes (NB), Logistic Regression(LR),  Decision Tree(DT), Support Vector Machine(SVM) ,
 Random Forest(RF), and Artificial Neural Network(ANN). We conducted experiments on five datasets about patients living in Senegal. The raw
 datasets have been collected in different settings and contain clinical data such as sign, symptom and the diagnostic of the doctor.  The outcome of the RDT is also provided. Our main result is that Random Forest, Logistic Regression, Support Vector Machine with Gaussian kernel and Artificial Neural Network outperforms RDT and present very high precision in the Senegalese patient datasets.
The rest of the paper is organized as follows. We start by presenting the methods used in this work in section \ref{Methods}. In section \ref{results_discussion} details the results of the intensive experimentations conducted over various datasets. Finally, we conclude this paper in section \ref{conclusion}.
