\section{Experimentation and results}\label{experimentations}
We detail and analyze in the section the results of the experimentation we performed using the six ML algorithms presented in Section \ref{ml_algorithms} over the two real datasets described in Section \ref{datasets}. We start by presenting our experimentation setting.

% experimentation setting
\subsection{Experimentation Setting}
In this section data is available for applying classification algorithm. After model creation from training data, classification operation is performed on test data. 
All the performed tests have been done in the same machine and the same operating system. To test the performance of our six chosen ML algorithms, we relied on their Python implementations available through the scikit-learn library6. Scikit-learn is an open source simple and efficient tool for predictive data analysis that implements most of the existing ML algorithms

Then some of the most important performance evaluation measures like accuracy, precision, sensitivity, specificity, F-measure and area under ROC curve are evaluated and compared. 
For the details about the description of each parameter of ML we refer to the official documentation of the implementation of these algorithms in scikit-learn7. Concerning the segmentation of both datasets for the training of our ML algorithms and their testing we have considered the stratified-5-fold cross-validation in classification model construction and efficiency evaluation. This method is very useful to handle data with an unbalanced class distribution, increases the validation of classification and prevents from random and invalid results.


% Results of the experiments
\subsection{Results of the experiments}
This section presents the results of the experimentation on each real dataset for each of the six classifiers. 


\subsubsection*{\bf Experiments with DT1.}
