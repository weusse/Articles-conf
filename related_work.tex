\section{relate dwork}\label{literature}
We mainly provide a brief overview of Machine Learning and Deep Learning models for healthcare applications. 

In healthcare, Machine Learning apps can help better understand each patient's care journey, medical decisions, or the impact of new drugs. The survey of \cite{tomar2013survey} explores the usefulness of various data mining techniques such as classification, grouping, association, regression in the health field. This survey also highlights the applications, challenges and future issues of data mining in healthcare. The recommendation regarding the appropriate choice of available data mining technique is also discussed in this article. The authors of [2] explain the fundamental principles of logistic regression and the stages of its application. Using two examples (the quality of follow-up care for diabetics and hospital mortality after acute myocardial infarction), they demonstrate the value that this statistical tool can have in studies carried out by the medical service of the national health fund, especially in studies aimed at evaluating professional practice. Another example of previous work that has used logistic regression is that of Farida et al. \cite{adimi2010towards}. The logistic regression is exploited there for the selection of features in order to construct stable decision trees. The decision trees are then used to predict the severity criteria of Malaria in the context of Afghanistan.  
In [18] Uddin et al, provide a broad overview of the relative performance of different variants of supervised machine learning algorithms for disease prediction. Thus, their results showed that the Support Vector Machine (SVM) algorithm is applied most frequently (in 29 studies) followed by the Naïve Bayes algorithm (in 23 studies). However, the Random Forest (RF) algorithm has shown comparatively higher accuracy. Of the 17 studies where it has been applied, RF has shown the highest accuracy in 9 of them, or 53\%. This was followed by SVM which exceeded 41\% of the studies it considered.\cite{de2018binary} Presents a reference of 7 machine learning algorithms used on binary classification tasks and applied to hospital data. In the study \cite{Ug1} Data mining acts as a solution to many health problems and it is useful for predicting early stage heart disease. The Naive Bayes algorithm is one such data mining technique that helps predict heart disease in patients. Gharehchopogh et al. \cite{gharehchopogh2012application} explain the use of medical data mining in determining methods of medical operation. They show that the decision tree algorithm designed for this case study generates a correct prediction for more than 86\% of test cases.
Indeed, decision trees based approach has been proposed in Nigeria \cite {ugwu2012application} to predict the occurrence of Malaria given diagnostic data. In the same line of works applying machine learning, in \cite{rajpurkar2017malaria}, Pranav et al. propose Malaria likelihood prediction model built on a deep reinforcement learning (RL) agent. Such a RL predicts the probability of a patient testing positive for Malaria using answers from questions about their household. In the presented approach the authors have also dealt with the problem of determining the right question to ask next as well as the length of the survey, dynamically.
\subsection{Description des données}

