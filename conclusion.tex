\section{Conclusion}\label{conclusion}
In this study, six classifiers using a wide variety of operating procedures have been extensively tested and compared over real world health datasets in order to evaluate their performance for the task of predicting the occurrence or not of Malaria in a patient knowing his signs and symptoms. The results obtained show that the algorithms RF, LR,
SVM with Gaussian kernel and ANN present the best performances in predicting the occurrence or not of Malaria. In addition those four algorithms outperform the Rapid Diagnosis Test which is the standard diagnostic tool largely adopted in the health system in
Senegal. This research has indicated that in practice there is no single best classification tool, but instead the best technique will on the characteristics of the dataset to be analysed. 
Future work consists in the study and the implementation of an ensemble method for predicting the occurrence or not of malaria based on the classifiers offering the best performances in our present study. But also to compare these performances with the ensemble methods for their validation
