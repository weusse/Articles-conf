\section{Conclusion}\label{conclusion}
In this paper we have studied the problem of predicting the occurrence or not of Malaria given ill-patient dataset in the context of Senegal and by using machine learning techniques.
To tackle this problem we have first presented a data preparation pipeline that enables to clean, normalize and impute missing values
given a real-world dataset using efficient tools and algorithms. We also introduced a manner to extract the features that characterize the Malaria disease.
We have then proposed a prediction model based on the logistic regression to determine the occurrence of Malaria. The performance of such a model has been
demonstrated through extensive experimentations on real-world and semi-synthetic datasets. As a research perspective we plan to first include a prevalence 
factor into our prediction function in order to improve its accuracy. Second, we will use other binary classification models such as Support Vector Machine
(or SVM in short) and compare their results to those obtained with the logistic regression based model.
