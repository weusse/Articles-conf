% This is samplepaper.tex, a sample chapter demonstrating the
% LLNCS macro package for Springer Computer Science proceedings;
% Version 2.20 of 2017/10/04
%
%\documentclass[runningheads]{llncs}
%
\documentclass[10pt,a4paper]{article}
\usepackage{hyperref}
\renewcommand\UrlFont{\color{blue}\rmfamily}
\setlength{\parskip}{0pt}
\raggedbottom
\usepackage{multirow}
\usepackage{amsmath}
\usepackage{csquotes}
\usepackage{paralist}
\usepackage{booktabs}
\usepackage{mathptmx}
\usepackage{pgfplots}
\pgfplotsset{compat=1.8}
\usepackage{subfigure}
\usepackage{pgfplots}
\pgfplotsset{width=7cm,compat=1.8}
\usepackage{pgfplotstable}
\renewcommand*{\familydefault}{\sfdefault}
\usepackage{tikz}
%\pgfplotset{compact=1.16 }
\begin{document}
\begin{table}[h]
\caption{Rendement du blé et des plants de tomates (g.pot$^{-1}$) pour différentes teneurs en uranium appliquées et différents niveaux d'irrigation (d'après~\cite{gulati1980})} \label{tablehormesis}
\begin{center}
\begin{tabular}{l|ccccc}
\toprule[.4mm]
                        & \multicolumn{4}{c}{Teneur en uranium appliquée (en ppm)} \\
Irrigation (L)          &  0   & 1,5  & 3,0  & 6,0  \\\midrule[.4mm]
Blé                     &      &      &      &      \\
\makebox[3cm][c]{11,52} & 10,2 & 11,7 & 14,3 & 12,8 \\
\makebox[3cm][c]{14,40} & 12,7 & 12,9 & 15,3 & 13,3 \\
\makebox[3cm][c]{19,20} & 13,0 & 12,7 & 12,0 & 11,9 \\
                        &      &      &      &      \\
Tomate                  &      &      &      &      \\
\makebox[3cm][c]{15,12} & 160  & 125  & 120  & 115  \\
\makebox[3cm][c]{18,90} & 182  & 167  & 165  & 145  \\
\makebox[3cm][c]{25,20} & 200  & 198  & 170  & 145  \\
\bottomrule[.4mm]
\end{tabular}
\end{center} 
\end{table}
\end{document}
