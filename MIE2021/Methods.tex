\section{Methods}\label{Methods}
In this part we discusse about the methods and the technic of machine learning used in this study
\subsection{Methodology}

\subsection{Machine Learning algorithms}
In the following we discuss about some of these methods.  Those algorithms are chosen among the most used ones in the health field according to studies\cite{de2018binary,tomar2013survey}.\\
\textbf{Decision tree (DT)}\cite{Ro05} is a supervised classifier which is obtained by recursively partitioning the labelled set of observations. It is one of the most adopted classifiers, thanks to its simplicity and its straightforward interpretation. For CART algorithms, hyperparameters are the impurity criteria (entropy and gini), the maximum depth, the minimum samples to split and the minimum samples at a leaf

% Random forest
\textbf{Random Forest (RF)}\cite{Be01} is an ensemble approach built upon many decision tree classifiers. It is a supervised classifier which requires the same hyper parameters as DT, plus the number of trees to create and the random number of features to look at when splitting the labelled data during the training step \cite{Be01}.\\
% Naive Bayes
\textbf{ Naive Bayes classifier (NB) }\cite{Ka17} is a\emph{supervised} machine learning algorithm, i.e. requires to be trained, used for classifying observations to given distinct classes based on \emph{input explanatory variables} (a.k.a feature or attribute).
It is a classification technique based on the well-known \emph{Bayes’ theorem}\footnote{https://en.wikipedia.org/wiki/Bayes\%27\_theorem} with strong and naive assumptions. It simplifies learning by assuming that features are independent of given class.\\
% Logistic Regression 
\textbf{Logistic regression (LR)} \cite{Ph88} is a statistical model used in the machine learning domain as a supervised classifier for binary classification \cite{uddin2019comparing}. 
It is based, in its basic form, on a logistic function to describe a binary dependent variable\cite{wang2014support,de2018binary} by considering as input 
qualitative or/and ordinal explanatory variables  in order to measure the probability of a given class label. The greatest advantage  of the logistic regression
classifier is the fact that you can use continuous explanatory variables and it is easier to handle more than two explanatory variables simultaneously and its ability to quantify the strength of the relationship between each explicative variable and the variable to explain, given the other variables integrated to the model.\\
%Support Vector Machine
\textbf{Support Vector Machine (SVM)} \cite{Ev01} is a supervised classification approach whose intuition is to represent input data in a space and to determine the optimal hyper-plane that divides that space in two regions depending on the targeted value.\\
% artificial neural networks
\textbf{An Artificial Neural Network (ANN)} \cite{Me19} is a computational approach also referred to as a Connectionist System used in Machine Learning. ANNs are loosely modeled after the biological neural network in an attempt to replicate the way in which we learn as humans. Think of it as a computing system, structured as a series of layers, each layer consisting of one or several neurons. The types of the layers comprise \emph{input}, \emph{output} and \emph{hidden} layers \cite{anderson1972simple,raschka2015python}.



